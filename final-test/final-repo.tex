\documentclass[a4j]{jarticle}

\usepackage{url}

\usepackage{ascmac}

\textwidth=16cm

\oddsidemargin=0cm

\title{情報工学実験C コンパイラ}

\author{氏名:大西 隼也 \\学籍番号:09427510}

\date{出題日: 2017年12月5日\\提出日:2018年2月6日 \\締切日:2018年2月6日}

\begin{document}


\maketitle


\section{実験の目的}
本実験では,プログラミング言語(高級言語)で書かれたプログラムを入力し,コンピュータが実行できる言語(低級言語)に変換するプログラムであるコンパイラの簡易版を作成する.

実験の目的として,yacc,lexといったプログラムジェネレータ(プログラムを作成するプログラム)を用いる経験をすること,またコンパイルの作成を通じて,プログラミング言語で書かれたプログラムとアセンブリ言語との対応について理解を深めるというものがある.

\section{今回作成した言語の定義}

今回,作成したコンパイラでは,受理するプログラムの関係上,基本言語仕様の拡張が必要なかったため,そのまま用いている.

\begin{verbatim}
<プログラム> ::= <変数宣言部> <文集合>
<変数宣言部> ::= <宣言文> <変数宣言部> | <宣言文>
<宣言文> ::= define <識別子>; | array <識別子> [ <数> ];
<文集合> ::=  <文> <文集合>| <文>
<文> ::= <代入文> | <ループ文> | <条件分岐文>
<代入文> ::= <識別子> = <算術式>;
<算術式> ::= <算術式> <加減演算子> <項> | <項>
<項> ::= <項> <乗除演算子> <因子> | <因子>
<因子> ::= <変数> | (<算術式>)
<加減演算子> ::= + | -
<乗除演算子> ::= * | /
<変数> ::= <識別子> | <数> | <識別子> [ < <数> ]
<ループ文> ::= while (<条件式>) { <文集合> }
<条件分岐文> ::= if (<条件式>) { <文集合> } 
			 | if (<条件式>) { <文集合> } else { <文集合> } 
<条件式> ::= <算術式> <比較演算子> <算術式>
<比較演算子> ::= == | '<' | '>'
<識別子> ::= <英字> <英数字列> | <英字>
<英数字列> ::= <英数字> <英数字列>| <英数字>
<英数字> ::= <英字> | <数字>
<数> ::= <数字> <数> | <数字>
<英字> ::= a|b|c|d|e|f|g|h|i|j|k|l|m|n|o|p|q|r|s|t|u|v|w|x|y|z
		|A|B|C|D|E|F|G|H|I|J|K|L|M|N|O|P|Q|R|S|T|U|V|W|X|Y|Z
<数字> ::= 0|1|2|3|4|5|6|7|8|9
\end{verbatim}


\section{定義した言語で受理されるプログラムの例}
今回定義した言語では以下のような仕様を満たすプログラムを受理可能である.

\begin{screen}
\begin{itemize}
\item define を用いて変数の定義が可能である.
\item 定義した変数に適切な値を代入可能である.
\item 変数と数字を用いた単純な四則演算の算術式を使用できる.
\item while を用いたループ文を使用できる.
\end{itemize}
\end{screen}

以下,受理されるプログラムの例を示す.

\begin{itemize}


\item 最終課題1

1から10までの数の和
\begin{verbatim}
define i;
define sum;

sum = 0;
i = 1;
while(i < 11) {
   sum = sum + i;
   i = i + 1;
}
\end{verbatim}

\item 最終課題2

5の階乗の計算
\begin{verbatim}
define i;
define fact;

fact = 1;
i = 1;
while(i < 6) {
   fact = fact * i;
   i = i + 1;
}
\end{verbatim}

\end{itemize}

\section{コード生成の概略}

作成した抽象構文木(AST)を元に,コード生成を行う.コード生成は,主に次のような段階で行われる.
\begin{itemize}

\item レジスタ割り当て規則の決定
\item メモリの使用方法の決定
\item 記号表の作成
\item ASTの各ノードに対応するコードの出力
	\begin{itemize}
	\item 変数領域の確保
	\item 算術式
	\item 代入文
	\item ループ文
	\item 条件分岐文
	\item 関数呼び出し
	\end{itemize}
\end{itemize}

以下,各段階について詳細に述べる.

\subsection{レジスタ割り当て規則}

\subsection{メモリの使用方法}

\subsection{算術式のコード生成について}

\section{最終課題を解くために書いたプログラムの概要}

\subsection{抽象構文木の作成に関して}

\subsection{コード生成に関して}


\section{最終課題の実行結果}

\section{特に工夫した点についての説明}



\section{考察}
\subsection{算術式のコード生成の難しさ}

\section{実験を通しての感想}
ペアプログラミング


\section{コンパイラのソースプログラム}

今回,作成したプログラムのソースコードについて,主に基本言語仕様について記述したyaccとlexのプログラム2つと抽象構文木及びコード生成を行う関数を記述したc言語のプログラム1つの計3つに分かれており,非常に膨大なページ数となるためgithubへのリンクと,プログラム名を記載することで割愛する.

\url{https://github.com/Shunya-Onishi/compila/tree/master/final-test}

\begin{itemize}
\item C言語のプログラム:makeast.c 
\item Yaccのプログラム:edu4.y
\item Lexのプログラム:edu4.l 
\end{itemize}

\end{document}